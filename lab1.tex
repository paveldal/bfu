\documentclass[11pt]{article}
\usepackage{amsmath,amssymb,amsthm}
\usepackage{algorithm}
\usepackage[noend]{algpseudocode} 
\usepackage{indentfirst}

%---enable russian----

\usepackage[utf8]{inputenc}
\usepackage[russian]{babel}
\usepackage{fancyhdr}
\usepackage{amsthm}

% PROBABILITY SYMBOLS
\newcommand*\PROB\Pr 
\DeclareMathOperator*{\EXPECT}{\mathbb{E}}


% Sets, Rngs, ets 
\newcommand{\N}{{{\mathbb N}}}
\newcommand{\Z}{{{\mathbb Z}}}
\newcommand{\R}{{{\mathbb R}}}
\newcommand{\Zp}{\ints_p} % Integers modulo p
\newcommand{\Zq}{\ints_q} % Integers modulo q
\newcommand{\Zn}{\ints_N} % Integers modulo N

% Landau 
\newcommand{\bigO}{\mathcal{O}}
\newcommand*{\OLandau}{\bigO}
\newcommand*{\WLandau}{\Omega}
\newcommand*{\xOLandau}{\widetilde{\OLandau}}
\newcommand*{\xWLandau}{\widetilde{\WLandau}}
\newcommand*{\TLandau}{\Theta}
\newcommand*{\xTLandau}{\widetilde{\TLandau}}
\newcommand{\smallo}{o} %technically, an omicron
\newcommand{\softO}{\widetilde{\bigO}}
\newcommand{\wLandau}{\omega}
\newcommand{\negl}{\mathrm{negl}} 
\renewcommand{\headrulewidth}{0pt} %убирает черту

% Misc
\newcommand{\eps}{\varepsilon}
\newcommand{\inprod}[1]{\left\langle #1 \right\rangle}
\usepackage{amsmath}

% Page
\pagestyle{fancy}
\fancyhf{}
\renewcommand{\headrulewidth}{0pt}
\chead{\textbf{Примитивные корни и индексы}}
\rhead{\textbf{Часть 8}}
\lhead{\textbf{168}}

\begin{document}
\noindent $\text{НОД}(a,p)=1.$ Если $p \nmid x$, тогда Теорема Ферма допускает $x^{p-1} \equiv 1 \pmod{p}$, откуда
\begin{equation*}
    a^{{(p-1)}/2} \equiv (x^2)^{(p-1)/2} \equiv 1 \pmod{p}.
\end{equation*}
Таким образом $a$ не может быть примитивным корнем $p$ [если $p \mid x$, тогда $p \mid a$ и конечно $a^{p-1} \not\equiv 1 \pmod{p})$]. Кроме того $(-1)^2=1$, $-1$ не является примитивным корнем $p$ в случае, если $p-1>2$. 

\noindent \textbf{Пример 8-3} \\
Давайте воспользуемся различными методами этого раздела, чтобы найти целые числа $\phi(6)=2$ имеющие порядок 6 по модулю 31. Начнем с того, что мы знаем, что существуют
\begin{equation*}
    \phi(\phi(31))=\phi(30)=8
\end{equation*}
примитивные корни 31. Получение одного из них-это вопрос проб и ошибок. Поскольку $2^5 \equiv 1 \pmod{31}$, целое число 2 явно исключается. Нам не нужно искать слишком далеко, так как 3 оказывается примитивным корнем из 31. Заметим, что при вычислении интегральных степеней 3 нет необходимости выходить за пределы $3^{15}$; для порядка 3 Необходимо разделить $\phi(31)=30$ и вычислить
\begin{equation*}
    3^{15} \equiv (27)^5 \equiv (-4)^5 \equiv (-64)(16) \equiv -2(16) \equiv -1 \not\equiv 1 \pmod{31}
\end{equation*}
показывает, что его порядок больше 15. \\
Поскольку 3 является примитивным корнем из 31, любое целое число, которое относительно просто 31, конгруэнтно по модулю 31 целому числу $3^k$, где $1 \leqslant k \leqslant 30$. Теорема 8-3 утверждает, что порядок $3^k$ равен $30/НОД(k,30)$; это будет равно 6 тогда и только тогда,когда $\text{НОД} (k, 30)=5$. Значения $k$, для которых выполняется последнее равенство, равны $k=5$ и $k=25$. Таким образом, наша задача теперь сводится к оценке $3^5$ и $3^{25}$ по модулю 31. Простой расчет дает
\begin{gather*}
    3^5 \equiv (27)^9 \equiv (-4)^9 \equiv -36 \equiv 26 \pmod{31}, \\
    3^{25} \equiv (3^5)^5 \equiv (26)^5 \equiv (-5)^5 \equiv (-125)(25) \equiv -1(25) \equiv 6 \pmod{31},
\end{gather*}
так что 6 и 26-единственные целые числа, имеющие порядок 6 по модулю 31. \\
\begin{center}
    {\bf ПРОБЛЕМЫ 8.2}
\end{center}
\begin{enumerate}
    \item Если $p$ это нечетное простое число, докажите, что
    \begin{enumerate}
        \item единственными решениями $x^2 \equiv 1 \pmod{p}$ являются $1$ и $p-1$;
        \item соответствие переменной $x^{p-2} + \cdots + x^2 + x + 1 \equiv 0 \pmod{p}$ имеет ровно $p-2$ неконгруэнтных решений: $2, 3, \ldots , p-1.$
    \end{enumerate}
    
\newpage
\rhead{\textbf{169}}
\chead{\textbf{Примитивные корни для простых чисел}}
\lhead{\textbf{Раздел 8-2}}

    \item Убедитесь, что каждая из конгруэнций $x^2 \equiv 1 \pmod{15}$,$ x^2 \equiv -1 \pmod{65}$ и $x^2 \equiv -2 \pmod{33}$ имеет четыре несогласованных решения; следовательно, Теорема Лагранжа не должна выполняться, если модуль является составным числом.
    \item Определите все примитивные корни простых чисел $p = 17, 19,$ и $23$, выражая каждый из них как степень кого-то из корней.
    \item Учитывая, что $3$ - это примитивный корень из $43$, найдите
    \begin{enumerate}
        \item все положительные целые числа меньше $43$, имеющие порядок $6$ по модулю $43$;
        \item все положительные целые числа меньше $43$, имеющие порядок $21$ по модулю $43$.
    \end{enumerate}
    \item Найдите все положительные целые числа меньше $61$, имеющие порядок $4$ по модулю $61$.
    \item Предположив, что $r$ является примитивным корнем нечетного простого числа $p$, установите следующие факты:
    \begin{enumerate}
        \item Конгруэнтность $r^{(p-1)/2}\equiv -1 \pmod{p}$ сохраняется
        \item Если $r'$ это любой другой примитивный корень $p$, тогда $rr'$ не является примитивным корнем $p$. [\emph{Подсказка}: По частям (a), $(rr')^{(p-1)/2} \equiv 1 \pmod{p}.$]
        \item Если для целого числа $r'$ выполняется условие $rr' \equiv 1 \pmod{p}$, тогда $r'$ примитивный корень $p$.
    \end{enumerate}
    \item Для простого числа $p > 3$, докажите, что примитивные корни $p$ встречаются в парах $r,r'$ где $rr' \equiv 1 \pmod{p}$. [\emph{Подсказка}: Если $r$ примитивный корень $p$, рассмотрите целое число $r'=r^{p-2}$.]
    \item Пусть $r$ - примитивный корень нечетного простого числа $p$. Докажите:
    \begin{enumerate}
        \item если $p \equiv 1 \pmod{4}$, тогда $-r$ является также примитивным корнем $p$;
        \item если $p \equiv 3 \pmod{4}$, тогда $-r$ имеет порядок $(p-1)/2$ по модулю $p$.
    \end{enumerate}
    \item Приведите другое доказательство теоремы 5-3, показав, что если $r$ является примитивным корнем простого числа $p \equiv 1 \pmod{4}$, то $r^{(p-1) / 4}$ удовлетворяет квадратичной конгруэнтности $x^2 + 1 \equiv 0 \pmod{p}$
    \item Используйте тот факт, что каждое простое число $p$ имеет примитивный корень, чтобы дать другое доказательство теоремы Уилсона. [\emph{Подсказка}: если $p$ имеет примитивный корень $r$, то по теореме 8-4 $(p-1)! \equiv r^{1 + 2 + \cdots + (p-1)} \pmod{p}$]
    \item Если $p$ является простым числом, покажите, что произведение $\phi(p - 1)$ примитивных корней $p$ конгруэнтно по модулю $p$ к $(-1)^{\phi(p - 1)}$. [\emph{Подсказка}: Если $r$ является примитивным корнем $p$, то $r^k$ также примитивный корень $p$ при условии, что $\text{НОД}(k,p - 1) = 1$; теперь используем теорему 7-7.]
    \item Для нечетного простого числа $p$, убедитесь, что сумма
    \begin{equation*}
    1^n + 2^n + 3^n + \cdots + (p - 1)^n = 
        \begin{cases}
            0 &\text{$\pmod{p}$ if $(p - 1) \nmid n$}\\
            -1 &\text{$\pmod{p}$ if $(p - 1) \mid n$}
        \end{cases}
    \end{equation*}
    $\Bigg[$\emph{Подсказка}: Если $(p - 1) \nmid n$, и $r$ есть примитивный корень $p$, тогда сумма конкурентна по модулю $p$ к $1 + r^n + r^{2n} + \cdots + r^{(p-2)n} = \cfrac{r^{(p-1)/n}-1}{r^n-1}$$\Biggl]$
\end{enumerate}

\newpage
\chead{\textbf{Примитивные корни и индексы}}
\rhead{\textbf{Часть 8}}
\lhead{\textbf{170}}

\begin{tabular}{llll}
\textbf{8.3} \quad & \textbf{СОСТАВНЫЕ ЧИСЛА, ИМЕЮЩИЕ} \\ & \textbf{ПРИМИТИВНЫЕ КОРНИ}
\end{tabular}

Ранее мы видели, что $2$ - это примитивный корень из $9$, так что составные числа также могут обладать примитивными тутами. Следующим шагом нашей программы является определение всех составных чисел, для которых существуют примитивные корни. Некоторая информация содержится в следующих двух отрицательных результатах. \\
\textsc{Теорема 8-7.} \textsl{Для $k \geq 3$, целые числа $2^k$ не имеют примитвных корней.} \\
\emph{Доказательство}: По причинам, которые станут ясны позже, мы начнем с того, что покажем, что если $a$ - нечетное целое число, то для $k \geq 3$.
\begin{equation*}
    a^{2^{k-2}} \equiv 1 \pmod{2^{k}}.
\end{equation*}
Если $k = 3$, эта конгруэнтность становится $a^2 \equiv 1 \pmod{8}$, что, безусловно, верно (действительно, $1^2 \equiv 3^2 \equiv t^2 \equiv 7^2 \equiv 1 \pmod{8}$). Для $k > 3$ мы исходим из индукции на $k$. Предположим, что заявленная конгруэнтность справедлива для целого числа $k$; то есть $a^{2^{k-2}} \equiv 1 \pmod{2^k}$. Это эквивалентно уравнению \\
\begin{equation*}
    a^{2^{k-2}} = 1 + b2^{k},
\end{equation*}
где $b$ - целое число. Возведя в квадрат обе стороны, получим
\begin{align*}
    a^{2^{k-1}} = (a^{2^{k-2}})^2 =& 1 + 2(b2^k) + (b2^k)^2 \\ =& 1 + 2^{k+1}(b+b^{2}2^{k-1}) \\ \equiv& 1 \pmod{2^{k+1}},
\end{align*}
таким образом, утверждаемая конгруэнтность справедлива для $k + 1$ и, следовательно, для всех $k \geq 3$. \\
Теперь целые числа, которые относительно просты для $2^k$, являются именно нечетными целыми числами; кроме того, $\phi{2^{k}} = 2^{k-1}$. По тому, что только что было доказано, если $a$ - нечетное целое число и $k \geq 3$
\begin{equation*}
    a^{\phi(2^{k})/2} \equiv 1 \pmod{2^k}
\end{equation*}
и, следовательно, не существует примитивных корней $2^k$. \\
Следующая теорема в том же духе: \\
\textsc{Теорема 8-8.} \textsl{Если $\text{НОД}(m,n)=1$, где $m>2$ и $n>2$, тогда целое число $mn$ не имеет примитивных корней.}

\newpage
\rhead{\textbf{171}}
\chead{\textbf{Cоставные числа, имеющие \\ примитивные корни}}
\lhead{\textbf{Раздел 8-3}}

\emph{Доказательство}: Рассмотрим любое целое число $a$, для которого $\text{НОД}(a,mn)=1$; тогда $\text{НОД}(a,m)=1$ и $\text{НОД}(a,n)=1$. Положим $h=\text{НОК}(\phi(m),\phi(n))$ и $d=\text{НОД}(\phi(m),\phi(n)).$ \\
Поскольку $\phi(m)$ и $\phi (n)$ четны (Теорема 7-4), то, конечно, $d \geq 2$. Вследствие,
\begin{equation*}
    h = \frac{\phi(m)\phi(n)}{d} \leq \frac{\phi(mn)}{2}.
\end{equation*}
Теперь Теорема Эйлера утверждает, что $a^{\phi(m)} \equiv 1 \pmod{m}$. Поднимая это уравнение до степени $\phi (n)/d$, мы получаем
\begin{equation*}
    a^h = (a^{\phi(m)})^{\phi(n)/d} \equiv 1^{\phi(n)/d} \equiv 1 \pmod{m}
\end{equation*}
Аналогичное рассуждение приводит к $a^h \equiv 1 \pmod{n}$. Вместе с гипотезой $\text{НОД} (m,n)=1$ эти конгруэнции заставляют сделать вывод, что
\begin{equation*}
    a^h \equiv 1 \pmod{mn}.
\end{equation*}
Мы хотим подчеркнуть, что порядок любого целого числа относительно простого к $mn$ не превышает $\phi (mn)/2$, откуда не может быть никаких примитивных корней для $mn$. \\
Некоторые частные случаи теоремы 8-8 представляют особый интерес, и мы перечислим их ниже. \\
\textsc{Заключение.} \quad \textsl{Целое число n не имеет примитивного корня, Если либо}
\begin{enumerate}
\item n делится на два нечетных простых числа, или
\item n имеет вид $n = 2^{m}p^{k}$, где $p$ - нечетное простое число и $m \geq 2$.
\end{enumerate}
\qquad Существенной особенностью этой последней серии результатов является то, что они ограничивают наш поиск примитивных корней целыми числами $2, 4, p^k и 2p^k$, где $p$ - нечетное простое число. В этом разделе мы докажем, что каждое из только что упомянутых чисел имеет примитивный корень, главной задачей которого является установление существования примитивных корней для степеней нечетного простого числа. Этот спор несколько затянут, но в остальном рутинен; для ясности он разбит на несколько этапов. \\
\begin{center}
\textsc{Лемма 1.} \quad \textsl{Если $p$ - нечетное простое число, то существует примитивный корень $r$ из $p$ такой, что $r^{p-1} \not\equiv 1 \pmod{p^2}$.}
\end{center}
\emph{Подсказка}: Из теоремы 8-6 известно, что $p$ имеет примитивные корни. Выберите любой из них, назовите его $r$. Если $r^{p-1} \not\equiv 1 \pmod{p^2}$, то мы закончили.
\end{document}
